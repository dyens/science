\documentclass[12pt]{article}
\textwidth = 15cm \oddsidemargin = 1.5cm \topmargin = -3cm \textheight = 21cm
%%\usepackage[russian]{babel}
\usepackage{feynmf}
\usepackage{amsmath} %% подключение математического пакета
%\usepackage{epstopdf}
%\usepackage{graphicx}
\usepackage{graphics}
\unitlength=1mm
%\documentstyle[12pt]{article}
%\renewcommand{\textheight}{22 cm}
%\renewcommand{\textwidth}{16 cm}
%\oddsidemargin 0cm \addtolength\topmargin{-2cm}

% DEFINITIONS
\def\eps{\varepsilon}
\def\epsilon{\varepsilon}
\def\Dm{\widetilde{\cal D}_{\mu}}
\def\const{{\rm const\,}}
\def\D{{\cal D}}
\def\L{{\cal L}}
\def\A{{\cal A}}
\def\S{{\cal S}}
\def\x{{\bf x}}
\def\p{{\bf p}}
\def\k{{\bf k}}
\def\q{{\bf q}}
\def\r{{\bf r}}
\def\h{{\bf h}}
\def\n{{\bf n}}
\def\bfx{{\bf x}}
\def\bfv{{\bf v}}
\def\bfu{{\bf u}}
% END OF DEFINITIONS

\begin{document}
%\section{Introduction} 
%\label{sec:Intro}

\section{Description of the model. Field theoretic formulation}
test
\label{sec:QFT}
\begin{equation}
    S_{v}(v^{'},v)=v^{'}D_{v}v^{'}/2+v^{'}\{-\nabla_{t}+\nu_{0}\partial^2\}v+\omega_{0}\nu_{0}^{3}v^{'}_{i}(\partial_{i}\theta)\partial^{2}\theta
\end{equation}
\begin{equation}
S(v,v^{'},\theta,\theta^{'}) = S_{v}(v^{'},v)+\theta^{'}D_{\theta}\theta^{'}/2+\theta^{'}\{-\nabla_{t}+k_{0}\partial^{2}\}\theta
\end{equation}
Feynman rules.
\begin{equation}
\langle v,v^{'} \rangle_{0}=\langle v^{'},v \rangle _{0}^{T}=(-i\omega+\nu_{0}k^2)^{-1}
\end{equation}
\begin{equation}
\langle v,v \rangle_{0}=P_{ij}D_{0}k^{4-d-y}|-i\omega+\nu_{0}k^2|^{-2}
\end{equation}
\begin{equation}
\langle \theta,\theta^{'} \rangle_{0}=\langle \theta^{'},\theta \rangle _{0}^{T}=(-i\omega+\nu_{0}k^2)^{-1}
\end{equation}
\begin{equation}
\langle \theta,\theta \rangle_{0}=K_{0}k^{2-d-y}|-i\omega+\nu_{0}k^2|^{-2}
\end{equation}
\begin{equation}
    \langle v_{i}^{'}, \theta,\theta \rangle_{0}= -i\omega_{0}\nu_{0}^{3}[p_i k^2+p^2 k_i]
\end{equation}
p,k - momentum of field $\theta$.
\begin{equation}
\langle \theta^{'}, v_i ,\theta \rangle_{0}= -ip_i
\end{equation}
p - momentum of field $\theta^{'}$, because $\partial_i v_i=0$.
\begin{equation}
\langle v^{'}_i, v_k ,v_s \rangle_{0}= -i[p_k \delta_{is}+ p_s \delta_{ik}]
\end{equation}
p - momentum of field $v^{'}$.
\section{Canonical dimensions, UV divergences and the renormalization}
\label{sec:Reno}

%\begin{tabular}{|c|c|c|c|c|c|c|c|c|c|c|}
%\hline
%$F$          & $\psi$ & $\psi^\dagger$ & $v$ & $\lambda_0,\lambda$ & $\tau_0,\tau$ & $ m,\mu,\Lambda$ & $g_0^2$ & $\omega_0$ & $g^2,\omega,\alpha,a_0,a$ \\
%\hline
%$d_F^k$      & $\frac{d-2}{2}$ & $\frac{d+2}{2}$ & $-1$ & $-2$ & $2$ & $1$ & $2-d$ & $\xi$ & $0$ \\
%\hline
%$d_F^\omega$ & 0 & 0 & $1$ & $1$ & $0$ & $0$ & $2$ & $0$ & $0$ \\
%\hline
%$d_F$        & $\frac{d-2}{2}$ & $\frac{d+2}{2}$ & $1$ & $0$ & $2$ & $1$ & $6-d$ & $\xi$ & $0$ \\
%\hline
%\end{tabular}

\begin{tabular} {|c|c|c|c|c|c|c|c|c|c|c|}
    \hline
    $F$            &  $v$   &   $v^{'}$  &   $\theta$            &     $\theta^{'}$            &        $\nu$     &     $\omega, \omega_{0}$    &      $D,D_{0}$     &   $k,k_{0}$      &   $g,g_{0}$       &   $u$    \\ \hline
    $d_F^k$        &  $-1$  &   $d+1$    &   $-\frac{1}{2}z+1$   &     $d+\frac{1}{2}z-1$      &        $-2$      &     $z$                     &      $y-6$         &   $-2$           &   $y$             &    0     \\ \hline
    $d_F^{\omega}$ &  $1$   &   $-1$     &   $-\frac{1}{2}$      &     $\frac{1}{2}$           &        $1$       &     $0$                     &      $3$           &   $1$            &   $0$             &    0     \\ \hline
    $d_F$          &  $1$   &   $d-1$    &   $-\frac{1}{2}z$     &     $d+\frac{1}{2}z$        &        $0$       &     $z$                     &      $y$           &   $0$            &   $y$             &   $0$    \\ \hline
\end{tabular}

The role of the coupling constants is played by the parameters $g_0 = D/\nu^3_0, u_0 = k_0/\nu_0$, and $\omega_0$. The model is logarithmic at  $y = z = 0$.


\end{document}
