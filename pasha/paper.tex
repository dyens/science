\documentclass[12pt]{article}
\textwidth = 15cm \oddsidemargin = 1.5cm \topmargin = -3cm \textheight = 21cm
\usepackage[russian]{babel}
\usepackage{amsmath} %% подключение математического пакета
\unitlength=1mm
%\documentstyle[12pt]{article}
%\renewcommand{\textheight}{22 cm}
%\renewcommand{\textwidth}{16 cm}
%\oddsidemargin 0cm \addtolength\topmargin{-2cm}
\begin{document}
Информационно-логическая модель и программная реализация  информационно-аналитического комплекса
 «Прием в высшее учебное заведение»

Статья посвящена формализации предметной области и автоматизации процесса приема в  высшее учебное заведение России (уровень федерального или национального исследовательского университета). В статье в соответствии с законодательством России, регламентирующим процесс приема в высшее учебное заведение, политике ВУЗа в области реализации учебного процесса и выполнении контрольных цифр приема, проводится анализ предметной области – приема абитуриентов, строится информационно-логическая модель процесса приема и анализируется программная реализация разрабатываемого информационно-аналитического комплекса.
 	В соответствии с новыми целями и задачами по автоматизации процесса приема в университет решалась задача создания расширенного программного комплекса, который включал бы в себя не только информационно-аналитическую обработку данных по приему, но и весь процесс приема абитуриентов с использованием современных информационных технологий.
Методологической основой информационно-аналитического обеспечения разрабатываемого  расширенного программного комплекса, является методология информационно-аналитического программного комплекса [1]. 
При автоматизации проведении приема в Санкт-Петербургский государственный университет (СПбГУ) руководство особое внимание уделяло вопросам предоставления открытой информации, удобства ввода данных для абитуриентов при поступлении, оперативности обработки информации и полноты получения отчетов. При этом согласно техническому заданию  разрабатываемый программный комплекс приема является составной частью создаваемого общего программного комплекса, включающего в себя прием абитуриентов, учебный процесс, внеучебную работу студентов и трудоустройство выпускников. 
В основу выбора программных средств по реализации программного комплекса приема было выработано идеология об использовании таких средств, чтобы они были удобны в использовании для абитуриентов, работников приемной комиссии, руководителей университета разного уровня. Применяемые информационные технологии должны быть доступны для применения всеми лицами, участвующими в процессе приема, удобны при использовании, надежны при работе и работать в реальном режиме времени. 
В этой статье, в соответствие с изложенными выше принципами, рассматриваются следующие аспекты разработки расширенного программного информационно-аналитического комплекса:
анализ сущностей предметной области приема в ВУЗ, описание  функционала решаемых задач с учетом специфики Санкт-Петербургского государственного университета;
выбор и обоснование использования аппаратных средств;
использование соответствующих информационных технологий и программных средств создания программного комплекса; 
разработка развитой системы администрирования и прав доступа для пользователей разного уровня;
вопросы импорта и экспорта данных;
информационно-аналитический блок обработки данных, математические методы обработки информации.

Анализ и формализация предметной области  -- прием в СПбГУ
В соответствие с Законом (о двух университетах) и программой развития Санкт-Петербургского государственного университета, основной задачей СПбГУ является обучение и подготовка кадров высшей квалификации – магистров, аспирантов и докторантов, проведение научных исследований, привлечение в СПбГУ через олимпиадное движение талантливой молодежи. Таким образом, наряду с традиционными задачами приема абитуриентов на первый курс бакалавриата и специалитета, возникает еще ряд специфических задач по приему в магистратуру. Например, наличие различных форм для поступления в магистратуру, различные виды магистерских программ – с двойным дипломом, включенном образовании т.д. Приемная комиссия СПбГУ уделяет большое внимание вопросам анализа контингента поступающих, проведению профориетационной работы и олимпиад. В программном комплексе все информационные потоки данных рассматриваются в разрезах: 
университета;
факультетов;
направления подготовки;
формы и основы обучения;
предметов вступительных экзаменов;
регионов;
олимпиад.

Вся импортируемая и вводимая в систему информация записывается в соответствующих таблицах базы данных. Далее информация обрабатывается с пом

Аппаратные средства реализации программного комплекса
          СПбГУ крупный образовательный центр, в его состав входит порядка 24 факультетов, на которых обучаются более 20 тыс. студентов. Ежегодно на программы бакалавриата и подготовки специалистов поступает более 4 тыс. человек и около 2 тыс. поступает на программы магистратуры. В период приемной компании размещение комиссий по приему документов происходит в разных зданиях, поэтому для эффективной работы требуется надежная и защищенная сеть. Для организации сетевого окружения выбрана базовая топология компьютерной сети – активная звезда. Как для любого программного обеспечения такого уровня, выдвигается ряд требований по обеспечения безопасности хранения персональных данных, согласно 94-Федеральному закону "О защите персональных данных". СПбГУ имеет аттестованную ФСТЭК локальную компьютерную сеть. Для хранения данных используется база данных MS SQL 2008. Программное обеспечение написано на .NET.
Характеристики используемого сервера для хранения базы данных: Microsoft Windows Server 2008 R2 Enterprise 64bit, 8 ядерный процессор 4xIntel Xeon E7-L8867 @ 2.13 GHz, 24 Гб оперативной памяти, Microsoft SQL Server Standard Edition (64-bit).
Характеристики используемого терминального сервера: Microsoft Windows Server 2008 R2 Enterprise 64bit, 4 ядерный процессор 2хIntel Xeon E7-L8867 @ 2.13 GHz, 128 Гб оперативной памяти.

Информационные технологии и программные решения
Для решения поставленной перед нами задачи было решено использо-вать клиент-серверную архитектуру, это позволяет перенести всю нагрузку с клиентских машин на мощные серверы. Понятие архитектуры клиент-сервер в системах управления предприятием связано с делением любой прикладной программы на три основных компонента. Этими тремя компонентами являются:
компонент представления (визуализации) данных;
компонент прикладной логики;
компонент управления базой данных. (http://belani.narod.ru/1/Lklser2.htm)

В нашем случае в качестве клиента используется терминальный сервер.  Такое решение позволило. Использовать для пользователей менее мощные рабочие станции, что экономит бюджет СПбГУ.
Удобный интерфейс для экспорт данных в любой формат
Возможность импорта данный по заданному шаблону


Администрирование программного комплекса
Личный кабинет поступающего, создан для удобного заполнения заявления поступающим. На данный момент для поступающих на программы магистратуры существует возможность подачи заявления в электронной форме с приложением полного сканированного комплекта документов. Введение данного модуля позволило значительно сократить время внесения данных о поступающих в ИС "Прием", теперь приемная комиссия только проверяет введенные поступающим данный. Так же для удобства поступающих в зданиях, где расположены комиссии по приему документов разворачивается компьютерный класс общего доступа, куда может прийти поступающий и заполнить заявление, если возникают сложности у поступающего, ему на помощь приходят волонтеры из числа студентов СПбГУ.
Права доступа - существует несколько ролей:
1. Технический секретарь - первичный ввод и проверка данных введенных поступающими, формирование личного дела поступающего;
2. Ответственный секретарь - общая организация работы комиссии по приему документов, создание протоколов о допуске, проверка личных дел, первичная проверка поступающих по льготному типу конкурса.
3. Советник проректора по направлениям - общий контроль работы комиссий по приему документов, допуск до участия в конкурсе поступающих по льготному типу конкурса.
4. Ректорат - просмотр общей информации о поступающих, просмотр статистических отчетов.
5. Шифровальная группа - шифрование и расшифровка экзаменационных работ поступающих, в случае проведения экзаменов в ВУЗе.
6. Начальник шифровальной группы - организация работы шифровальной группы, проверка внесенных данный, разнесение оценок в карточки поступающих.
7. Администратор - общая настройка системы, внесение данных о структуре приемной компании.
Иерархия отчетов - существует 4 уровня потребителей статистических отчетов:
1. Отчеты для внешних пользователей (правительства РФ, министерства образования и органов местного самоуправления) - такого вида отчеты передаются с определенной периодичностью, ежегодно повторяющиеся;
2. Отчеты для ректората - оперативная сводка о ходе приемной компании, отчеты о конкурсной ситуации и т.д.
3. Отчеты для комиссии по приему документов - ограничены только определенным направлением подготовки.
4. Отчеты для поступающих - сведения о конкурсной ситуации, сведения о дате подачи документов, рейтинговый список формируемый по окончании приемной компании.

Информационно-аналитический и математический модули программного комплекса
Информационно-аналитический и математический модули программного комплекса включают в себя математическую обработку данных, аналитику, визуализацию отчетные формы на трех уровнях. Первый уровень – это статистический анализ  исходных данных, получение основных числовых характеристик рассматриваемых сущностей.   Для более детальной статистической обработки данных в нужных разрезах на основе многокритериального отбора генерируются соответствующие отчеты. Для наглядного представления обработанных данных используются средства визуального отображения данных, диаграммы и графики.
Второй уровень – определение зависимостей, характеризующих   рассматриваемые данные с использованием системы статистических коэффициентов связи, идентификации законов распределения,  проведения анализа однородности данных, оценки параметров распределения и их изменений в зависимости от факультета и условий приема. Основным методом математического исследования на втором уровне является использование методологии применения кривых Пирсона и совокупности коэффициентов, описывающих  вероятностную связь между рассматриваемыми случайными величинами. 
Третий уровень – получение оценок сводных показателей, качественный анализ данных и подготовка принятия объективных управленческих решений.
Особое внимание при создании математического обеспечения программного комплекса уделяется вопросам нахождения оценок изменения законов распределения и параметров распределения в зависимости от выбранных факторов, например таких, как формы  и основы обучения. Также большое внимание уделялось изучению законов  распределения баллов ЕГЭ на различных факультетах и по различным предметам, оценкам влияния дополнительных факторов (победы в олимпиадах) и других факторов  на успеваемость.




Литература

1. Информационно-логическая модель и программная реализация  информационно-аналитического комплекса
 «Прием в высшее учебное заведение»

 Статья посвящена формализации предметной области и автоматизации процесса приема в  высшее учебное заведение России (уровень федерального или национального исследовательского университета). В статье в соответствии с законодательством России, регламентирующим процесс приема в высшее учебное заведение, политике ВУЗа в области реализации учебного процесса и выполнении контрольных цифр приема, проводится анализ предметной области – приема абитуриентов, строится информационно-логическая модель процесса приема и анализируется программная реализация разрабатываемого информационно-аналитического комплекса.
    В соответствии с новыми целями и задачами по автоматизации процесса приема в университет решалась задача создания расширенного программного комплекса, который включал бы в себя не только информационно-аналитическую обработку данных по приему, но и весь процесс приема абитуриентов с использованием современных информационных технологий.
    Методологической основой информационно-аналитического обеспечения разрабатываемого  расширенного программного комплекса, является методология информационно-аналитического программного комплекса [1]. 
    При автоматизации проведении приема в Санкт-Петербургский государственный университет (СПбГУ) руководство особое внимание уделяло вопросам предоставления открытой информации, удобства ввода данных для абитуриентов при поступлении, оперативности обработки информации и полноты получения отчетов. При этом согласно техническому заданию  разрабатываемый программный комплекс приема является составной частью создаваемого общего программного комплекса, включающего в себя прием абитуриентов, учебный процесс, внеучебную работу студентов и трудоустройство выпускников. 
    В основу выбора программных средств по реализации программного комплекса приема было выработано идеология об использовании таких средств, чтобы они были удобны в использовании для абитуриентов, работников приемной комиссии, руководителей университета разного уровня. Применяемые информационные технологии должны быть доступны для применения всеми лицами, участвующими в процессе приема, удобны при использовании, надежны при работе и работать в реальном режиме времени. 
    В этой статье, в соответствие с изложенными выше принципами, рассматриваются следующие аспекты разработки расширенного программного информационно-аналитического комплекса:
    анализ сущностей предметной области приема в ВУЗ, описание  функционала решаемых задач с учетом специфики Санкт-Петербургского государственного университета;
    выбор и обоснование использования аппаратных средств;
    использование соответствующих информационных технологий и программных средств создания программного комплекса; 
    разработка развитой системы администрирования и прав доступа для пользователей разного уровня;
    вопросы импорта и экспорта данных;
    информационно-аналитический блок обработки данных, математические методы обработки информации.

    Анализ и формализация предметной области  -- прием в СПбГУ
    В соответствие с Законом (о двух университетах) и программой развития Санкт-Петербургского государственного университета, основной задачей СПбГУ является обучение и подготовка кадров высшей квалификации – магистров, аспирантов и докторантов, проведение научных исследований, привлечение в СПбГУ через олимпиадное движение талантливой молодежи. Таким образом, наряду с традиционными задачами приема абитуриентов на первый курс бакалавриата и специалитета, возникает еще ряд специфических задач по приему в магистратуру. Например, наличие различных форм для поступления в магистратуру, различные виды магистерских программ – с двойным дипломом, включенном образовании т.д. Приемная комиссия СПбГУ уделяет большое внимание вопросам анализа контингента поступающих, проведению профориетационной работы и олимпиад. В программном комплексе все информационные потоки данных рассматриваются в разрезах: 
    университета;
    факультетов;
    направления подготовки;
    формы и основы обучения;
    предметов вступительных экзаменов;
    регионов;
    олимпиад.

    Вся импортируемая и вводимая в систему информация записывается в соответствующих таблицах базы данных. Далее информация обрабатывается с пом

    Аппаратные средства реализации программного комплекса
              СПбГУ крупный образовательный центр, в его состав входит порядка 24 факультетов, на которых обучаются более 20 тыс. студентов. Ежегодно на программы бакалавриата и подготовки специалистов поступает более 4 тыс. человек и около 2 тыс. поступает на программы магистратуры. В период приемной компании размещение комиссий по приему документов происходит в разных зданиях, поэтому для эффективной работы требуется надежная и защищенная сеть. Для организации сетевого окружения выбрана базовая топология компьютерной сети – активная звезда. Как для любого программного обеспечения такого уровня, выдвигается ряд требований по обеспечения безопасности хранения персональных данных, согласно 94-Федеральному закону "О защите персональных данных". СПбГУ имеет аттестованную ФСТЭК локальную компьютерную сеть. Для хранения данных используется база данных MS SQL 2008. Программное обеспечение написано на .NET.
              Характеристики используемого сервера для хранения базы данных: Microsoft Windows Server 2008 R2 Enterprise 64bit, 8 ядерный процессор 4xIntel Xeon E7-L8867 @ 2.13 GHz, 24 Гб оперативной памяти, Microsoft SQL Server Standard Edition (64-bit).
              Характеристики используемого терминального сервера: Microsoft Windows Server 2008 R2 Enterprise 64bit, 4 ядерный процессор 2хIntel Xeon E7-L8867 @ 2.13 GHz, 128 Гб оперативной памяти.

              Информационные технологии и программные решения
              Для решения поставленной перед нами задачи было решено использо-вать клиент-серверную архитектуру, это позволяет перенести всю нагрузку с клиентских машин на мощные серверы. Понятие архитектуры клиент-сервер в системах управления предприятием связано с делением любой прикладной программы на три основных компонента. Этими тремя компонентами являются:
              компонент представления (визуализации) данных;
              компонент прикладной логики;
              компонент управления базой данных. (http://belani.narod.ru/1/Lklser2.htm)

              В нашем случае в качестве клиента используется терминальный сервер.  Такое решение позволило. Использовать для пользователей менее мощные рабочие станции, что экономит бюджет СПбГУ.
              Удобный интерфейс для экспорт данных в любой формат
              Возможность импорта данный по заданному шаблону


              Администрирование программного комплекса
              Личный кабинет поступающего, создан для удобного заполнения заявления поступающим. На данный момент для поступающих на программы магистратуры существует возможность подачи заявления в электронной форме с приложением полного сканированного комплекта документов. Введение данного модуля позволило значительно сократить время внесения данных о поступающих в ИС "Прием", теперь приемная комиссия только проверяет введенные поступающим данный. Так же для удобства поступающих в зданиях, где расположены комиссии по приему документов разворачивается компьютерный класс общего доступа, куда может прийти поступающий и заполнить заявление, если возникают сложности у поступающего, ему на помощь приходят волонтеры из числа студентов СПбГУ.
              Права доступа - существует несколько ролей:
              1. Технический секретарь - первичный ввод и проверка данных введенных поступающими, формирование личного дела поступающего;
              2. Ответственный секретарь - общая организация работы комиссии по приему документов, создание протоколов о допуске, проверка личных дел, первичная проверка поступающих по льготному типу конкурса.
              3. Советник проректора по направлениям - общий контроль работы комиссий по приему документов, допуск до участия в конкурсе поступающих по льготному типу конкурса.
              4. Ректорат - просмотр общей информации о поступающих, просмотр статистических отчетов.
              5. Шифровальная группа - шифрование и расшифровка экзаменационных работ поступающих, в случае проведения экзаменов в ВУЗе.
              6. Начальник шифровальной группы - организация работы шифровальной группы, проверка внесенных данный, разнесение оценок в карточки поступающих.
              7. Администратор - общая настройка системы, внесение данных о структуре приемной компании.
              Иерархия отчетов - существует 4 уровня потребителей статистических отчетов:
              1. Отчеты для внешних пользователей (правительства РФ, министерства образования и органов местного самоуправления) - такого вида отчеты передаются с определенной периодичностью, ежегодно повторяющиеся;
              2. Отчеты для ректората - оперативная сводка о ходе приемной компании, отчеты о конкурсной ситуации и т.д.
              3. Отчеты для комиссии по приему документов - ограничены только определенным направлением подготовки.
              4. Отчеты для поступающих - сведения о конкурсной ситуации, сведения о дате подачи документов, рейтинговый список формируемый по окончании приемной компании.

              Информационно-аналитический и математический модули программного комплекса
              Информационно-аналитический и математический модули программного комплекса включают в себя математическую обработку данных, аналитику, визуализацию отчетные формы на трех уровнях. Первый уровень – это статистический анализ  исходных данных, получение основных числовых характеристик рассматриваемых сущностей.   Для более детальной статистической обработки данных в нужных разрезах на основе многокритериального отбора генерируются соответствующие отчеты. Для наглядного представления обработанных данных используются средства визуального отображения данных, диаграммы и графики.
              Второй уровень – определение зависимостей, характеризующих   рассматриваемые данные с использованием системы статистических коэффициентов связи, идентификации законов распределения,  проведения анализа однородности данных, оценки параметров распределения и их изменений в зависимости от факультета и условий приема. Основным методом математического исследования на втором уровне является использование методологии применения кривых Пирсона и совокупности коэффициентов, описывающих  вероятностную связь между рассматриваемыми случайными величинами. 
              Третий уровень – получение оценок сводных показателей, качественный анализ данных и подготовка принятия объективных управленческих решений.
              Особое внимание при создании математического обеспечения программного комплекса уделяется вопросам нахождения оценок изменения законов распределения и параметров распределения в зависимости от выбранных факторов, например таких, как формы  и основы обучения. Также большое внимание уделялось изучению законов  распределения баллов ЕГЭ на различных факультетах и по различным предметам, оценкам влияния дополнительных факторов (победы в олимпиадах) и других факторов  на успеваемость.
\end{document}

