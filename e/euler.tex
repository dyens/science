\documentclass{article}
\usepackage[latin1]{inputenc}
\usepackage[T1]{fontenc}
\usepackage{lmodern}
\usepackage[dvips,a4paper,scale=0.9]{geometry}
\usepackage{mathrsfs}
\usepackage{pstricks-add}
\usepackage{animate}
\usepackage[frenchb]{babel}



\psset{algebraic,unit=3cm,yunit=0.5}


% Macro utilisant les macros \x et \y
% qui devront �tre d�finies au pr�alable.
% #1 = le pas (un nombre comme 0,75 ou 0,5 ou 0,25 ou 0,1)
% #2 = la couleur
% Cette macro :
%   1) d�finit \xnew = \x+pas et \ynew = \y*(1+pas) en notation PostScript
%   2) trace le segment allant de (\x;\y) � (\xnew;\ynew)
%   3) et red�finit \x et \y en respectivement \xnew et \ynew
\newcommand{\Ligne}[2]%
{%
  \xdef\xnew{\x \space #1 add}
  \xdef\ynew{\y \space 1 #1 add mul}
  \psline[linecolor=#2,arrowscale=0.7]{*-*}(! \x \space \y)(! \xnew \space \ynew)
  \xdef\x{\xnew}
  \xdef\y{\ynew}
}



% Cr�er le fichier euler.tln ouvert en �criture
\newwrite\euler
\immediate\openout\euler=euler.tln
% Puis, on peut ensuite �crire dessus
\immediate\write\euler{::0x0} % le "fond d'�cran"
%
\immediate\write\euler{::1x0} % l�gende pour un pas de 0.75
\multido{\n=5+1}{4}{\immediate\write\euler{::\n x0}} % pas de 0.75
%
\immediate\write\euler{::2x0} % l�gende pour un pas de 0.5
\multido{\n=9+1}{6}{\immediate\write\euler{::\n x0}} % pas de 0.5
%
\immediate\write\euler{::3x0} % l�gende pour un pas de 0.25
\multido{\n=15+1}{12}{\immediate\write\euler{:4:\n x0}} % pas de 0.25
%
\immediate\write\euler{::4x0} % l�gende pour un pas de 0.1
\multido{\n=27+1}{30}{\immediate\write\euler{:8:\n x0}} % pas de 0.1
% Fermeture du fichier
\immediate\closeout\euler



\begin{document}

\begin{center}
\LARGE Construction de la fonction exp par la
m�thode d'Euler
\end{center}
\vspace{1cm}

On se restreint � l'intervalle $[0\,;\,3]$


\begin{center}


\begin{animateinline}[
  controls,
  buttonsize=2em,
  begin={\begin{pspicture}(-1,-1)(3,10)},
  end={\end{pspicture}},
  timeline=euler.tln
]{2}
  % frame 0 : rep�re et fonction exponentielle
  \psframe[linewidth=2pt](-1,-1)(3,10)
  \psaxes[arrowscale=2,labels=none]{->}(0,0)(-1,-1)(3,10)
  \pnode(1.90,7){A} \pnode(1,8){B}
  \psline{->}(B)(A)
  \uput{2pt}[140](B){$\mathscr{C}$ : $y= \exp(x)$}
  \psplot[linewidth=3pt]{-1}{3}{2.71828^x}
\newframe
  % frame 1 : l�gende pour un pas de 0.75
  \rput[l](1.5,1){\textcolor{green}{Avec un pas de 0,75}}
\newframe
  % frame 2 : l�gende pour un pas de 0.5
  \rput[l](1.5,1.5){\textcolor{blue}{Avec un pas de 0,5}}
\newframe
  % frame 3 : l�gende pour un pas de 0.25
  \rput[l](1.5,2){\textcolor{red}{Avec un pas de 0,25}}
\newframe
  % frame 4 : l�gende pour un pas de 0.1  
  \rput[l](1.5,2.5){\textcolor{violet}{Avec un pas de 0,1}}
\newframe
  % 4 frames (de 5 � 8) : lignes pour un pas de 0.75
  \def\x{0} \def\y{1} % (0;1) est le point de d�part des lignes bris�es
  \multiframe{4}{}{\Ligne{0.75}{green}}   
\newframe
  % 6 frames (de 9 � 14) : lignes pour un pas de 0.5
  \def\x{0} \def\y{1} % (0;1) est le point de d�part des lignes bris�es
  \multiframe{6}{}{\Ligne{0.5}{blue}}  
\newframe
  % 12 frames (de 15 � 26) : lignes pour un pas de 0.25
  \def\x{0} \def\y{1} % (0;1) est le point de d�part des lignes bris�es
  \multiframe{12}{}{\Ligne{0.25}{red}}     
\newframe
  % 30 frames (de 27 � 56) : lignes pour un pas de 0.1
  \def\x{0} \def\y{1} % (0;1) est le point de d�part des lignes bris�es
  \multiframe{30}{}{\Ligne{0.1}{violet}} 
\end{animateinline}
\end{center}

\end{document}
